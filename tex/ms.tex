% !TeX root = ./ms.tex
\documentclass[modern]{aastex62}

% Load the corTeX style definitions
% !TeX root = ./ms.tex
% All the packages
\usepackage{url}
\usepackage{amsmath}
\usepackage{mathtools}
\usepackage{amssymb}
\usepackage{natbib}
\usepackage{graphicx}
\usepackage{calc}
\usepackage{etoolbox}
\usepackage{xspace}
\usepackage[T1]{fontenc} % https://tex.stackexchange.com/a/166791
\usepackage{textcomp}
\usepackage{ifxetex}
\ifxetex
  \usepackage{fontspec}
  \defaultfontfeatures{Extension = .otf}
\fi
\usepackage{fontawesome}
\usepackage{listings}
\usepackage{nicefrac}
%\usepackage{bm}
\usepackage{booktabs}
\usepackage{longtable}

% Shorthand for this paper
\newcommand{\starry}{\textsf{starry}\xspace}
\newcommand{\Python}{\textsf{Python}\xspace}
\newcommand{\xxx}[1]{{\color{red}#1}}
\newcommand{\quadquad}{\quad\quad\quad\quad}

% References to text content
\newcommand{\documentname}{\textsl{article}}
\newcommand{\figureref}[1]{\ref{fig:#1}}
\newcommand{\Figure}[1]{Figure~\figureref{#1}}
\newcommand{\figurelabel}[1]{\label{fig:#1}}
\renewcommand{\eqref}[1]{\ref{eq:#1}}
\newcommand{\Eq}[1]{Equation~(\eqref{#1})}
\newcommand{\eq}[1]{\Eq{#1}}
\newcommand{\eqalt}[1]{Equation~\eqref{#1}}

% Add code, proof, and animation hyperlinks
\definecolor{linkcolor}{rgb}{0.1216,0.4667,0.7059}
\definecolor{testpasscolor}{rgb}{0.13333333,0.5254902,0.22745098}
\definecolor{testfailcolor}{rgb}{0.79607843,0.14117647,0.19215686}
\newcommand{\codeicon}{{\color{linkcolor}\faFileCodeO}}
\newcommand{\prooficon}{{\color{linkcolor}\faPencilSquareO}}
\newcommand{\testpassicon}{{\color{testpasscolor}\faCheckCircle}}
\newcommand{\testfailicon}{{\color{testfailcolor}\faTimesCircle}}
\newcommand{\codelink}[1]{\href{https://github.com/rodluger/starry_process/blob/0f5cd85041405565c68a93eca39244838420c99d/tex/figures/#1.py}{\codeicon}\,\,}
\newcommand{\animlink}[1]{\href{https://github.com/rodluger/starry_process/blob/0f5cd85041405565c68a93eca39244838420c99d/tex/figures/#1.gif}{\animicon}\,\,}
\newcommand{\prooflink}[1]{\href{https://github.com/rodluger/starry_process/blob/0f5cd85041405565c68a93eca39244838420c99d/tex/tests/#1.py}{\raisebox{-0.1em}{\input{tests/#1.tex}}}}
\newcommand{\cilink}[1]{\href{https://dev.azure.com/rodluger/starry_process/_build}{#1}}


% Define a proof environment for open source equation proofs
\newtagform{eqtag}[]{(}{)}
\newcommand{\currentlabel}{None}
\newenvironment{proof}[1]{%
  \ifstrempty{#1}{%
    \renewtagform{eqtag}[]{\raisebox{-0.1em}{{\color{red}\faPencilSquareO}}\,(}{)}%
  }{%
    \renewtagform{eqtag}[]{\prooflink{#1}\,(}{)}%
  }%
  \usetagform{eqtag}%
  \renewcommand{\currentlabel}{#1}
  \align%
}{%
  \endalign%
  \renewtagform{eqtag}[]{(}{)}%
  \usetagform{eqtag}%
  \message{<<<\currentlabel: \theequation>>>}%
}

% Define the `oscaption` command for open source figure captions
\newcommand{\oscaption}[2]{\caption{#2 \codelink{#1}}}

% Code examples
\definecolor{codegreen}{rgb}{0,0.6,0}
\definecolor{codegray}{rgb}{0.5,0.5,0.5}
\definecolor{codepurple}{rgb}{0.58,0,0.82}
\definecolor{backcolour}{rgb}{0.95,0.95,0.95}
\lstdefinestyle{mystyle}{
  backgroundcolor=\color{backcolour},
  commentstyle=\color{codegreen},
  keywordstyle=\color{magenta},
  numberstyle=\tiny\color{codegray},
  stringstyle=\color{codepurple},
  basicstyle=\small\ttfamily,
  breakatwhitespace=false,
  breaklines=true,
  captionpos=b,
  keepspaces=true,
  numbers=left,
  numbersep=5pt,
  showspaces=false,
  showstringspaces=false,
  showtabs=false,
  tabsize=2,
  aboveskip=1em,
  belowskip=1em,
  keywords=[2]{map},
  keywordstyle=[2]{\color{black!80!black}},
  upquote=true
}
\lstset{style=mystyle}

% Typography obsessions
\setlength{\parindent}{3.0ex}
\renewcommand\quad{\hskip\fontdimen3\font}

% https://tex.stackexchange.com/a/184474
\usepackage{stackengine,scalerel}
\def\lnlam{\ThisStyle{\ensurestackMath{\stackon[-2.4\LMpt]{%
        \SavedStyle\lambda}{\kern-.5pt\kern\LMpt\rule{1\LMex}{.25pt+.15\LMpt}}}}}

% Load custom style
% Packages
\usepackage{xifthen}
\usepackage{stackengine}
\usepackage{tabstackengine}
\usepackage{array}
\usepackage{upgreek}
\usepackage[bbgreekl]{mathbbol}
\usepackage{afterpage}
\usepackage[bb=boondox]{mathalpha}

% Misc. macros
\newcommand{\LMAX}{15\xspace}

% Integrals
\newcommand{\dd}{\ensuremath{\text{d}}}

% Special functions
\newcommand{\sgn}{{\text{sgn}}}
\newcommand{\atantwo}{{\text{arctan2}}}

% Cartesian unit vectors
\newcommand{\xhat}{\ensuremath{\pmb{\hat{x}}}\xspace}
\newcommand{\yhat}{\ensuremath{\pmb{\hat{y}}}\xspace}
\newcommand{\zhat}{\ensuremath{\pmb{\hat{z}}}\xspace}

% Other
\DeclarePairedDelimiter\ceil{\lceil}{\rceil}
\DeclarePairedDelimiter\floor{\lfloor}{\rfloor}

% Inverse diagonal dots
\makeatletter
\def\Ddots{\mathinner{\mkern1mu\raise\p@
                \vbox{\kern7\p@\hbox{.}}\mkern2mu
                \raise4\p@\hbox{.}\mkern2mu\raise7\p@\hbox{.}\mkern1mu}}
\makeatother

% Imaginary unit
\DeclareFontFamily{U}{mathc}{}
\DeclareFontShape{U}{mathc}{m}{it}{<->s*[1.03] mathc10}{}
\DeclareMathAlphabet{\mathscr}{U}{mathc}{m}{it}
\DeclareMathOperator{\imag}{\mathscr{i}}

% Bibliography
\bibliographystyle{aasjournal}

\usepackage{etoolbox}
\makeatletter % we need to patch \env@cases that has @ in its name
\patchcmd{\env@cases}{\quad}{\qquad\qquad}{}{}
\makeatother

\usepackage{enumitem}

% Begin!
\begin{document}

% Title
\title{%
    \textbf{
        Interpretable Gaussian Processes for Stellar Light Curves
    }
}

% Author list
\author[0000-0002-0296-3826]{Rodrigo Luger}\altaffiliation{Flatiron Fellow}
\email{rluger@flatironinstitute.org}
\affil{Center~for~Computational~Astrophysics, Flatiron~Institute, New~York, NY}
\affil{Virtual~Planetary~Laboratory, University~of~Washington, Seattle, WA}
%

\keywords{methods: analytic}

% \begin{abstract}
%     Abstract here.
%     %
%     \href{https://github.com/rodluger/starry_process}{\color{linkcolor}\faGithub}
% \end{abstract}

%
\section{The Spot Expansion}
\label{sec:spot}
%
\subsection{Preliminaries}
%
We adopt the following expression for the spherical harmonic coefficient
of degree $l$ and order $m$ in the expansion of a spot at
$\theta = \varphi = 0$:
%
\begin{align}
    \label{eq:ylm0}
    y_{lm}(r, \delta) =
    \begin{cases}
        1 - \dfrac{\delta c r}{2 (1 + c r)}
         & l = m = 0    \\[2em]
        -\dfrac{\delta c r \left( 2 + c r \right)}
        {2 \sqrt{2l + 1} (1 + c r)^{l + 1}}
         & l > 0, m = 0 \\[2em]
        0
         & m \ne 0
    \end{cases}
\end{align}
%
where $\delta \in [-\infty, 1]$ is the spot contrast (the fractional
decrease in the brightness
at the center of the spot) and $r \in (0, 1]$ is the normalized spot radius.
The quantity $c \in (0, \infty)$ is a normalization constant for the radius
(more details below).
The expression in Equation~(\ref{eq:ylm0}) is convenient because it satisfies
four important properties:
%
\begin{enumerate}[itemsep=2pt,parsep=1pt,label=\textbf{\arabic*}]
    \item The surface intensity monotonically increases away from the spot center
    \item The surface intensity at the spot center is $1 - \delta$
    \item The surface intensity at the antipode of the spot center is unity
    \item The size of the spot increases monotonically with $r$
\end{enumerate}
%
These properties may be demonstrated by considering the expression for the
surface intensity at a given point $(\theta, \varphi)$:
%
\begin{align}
    \begin{split}
        I(\theta, \varphi)
        & =
        \sum\limits_{l=0}^\infty \sum\limits_{m=-l}^l
        y_{lm} Y_{lm}(\theta, \varphi) \\
        & = \sum\limits_{l=0}^\infty
        y_{l0} \sqrt{2l + 1} P_l(\cos\theta)
    \end{split}
\end{align}
%
where $P_l$ is the Legendre polynomial of degree $l$ and we have implicitly
assumed a normalization such that the integral of our expansion over
the unit sphere is $4\pi$.
Combining this with Equation~(\ref{eq:ylm0}) and rearranging, we may write
%
\begin{align}
    \label{eq:Igen}
    I(\theta, \varphi) =
    1 + \dfrac{\delta c r}{2}
    -
    \dfrac{\delta c r \left( 2 + c r \right)}{2 (1 + c r)}
    \sum\limits_{l=0}^\infty \left(\dfrac{1}{1 + c r}\right)^l P_l(\cos\theta)
    \quad.
\end{align}
%
The summation in Equation~(\ref{eq:Igen}) has a closed-form expression in
terms of the generating function of the Legendre polynomials:
%
\begin{align}
    \label{eq:gen}
    \sum\limits_{l=0}^\infty t^l P_l(\cos\theta) = \frac{1}{\sqrt{1 + t^2 - 2 t \cos\theta}}
    \quad,
\end{align}
%
so we may express the intensity in the fairly simple form
%
\begin{align}
    \label{eq:Ifinal}
    I(\theta, \varphi) & = A - \frac{B}{\sqrt{C - \cos\theta}}
    \quad,
    %
    \\
    \intertext{where}
    %
    \begin{split}
        A & = 1 + \dfrac{\delta c r}{2}                      \\
        B & = \delta c r (2 + cr) \sqrt{\dfrac{1}{8 + 8 cr}} \\
        C & = \dfrac{1 + (1 + c r)^2}{2 + 2 c r}
    \end{split}
\end{align}
%
are positive constants.

Differentiating Equation~(\ref{eq:Ifinal}) with respect to $\theta$, we have
%
\begin{align}
    \label{eq:Ideriv}
    \dfrac{\mathrm{d}I(\theta, \varphi)}{\mathrm{d}\theta} & =
    -\frac{B\sin\theta}{2(C - \cos\theta)^\frac{3}{2}}
    \quad,
\end{align}
%
which is zero only for $\theta = 0$ (for which $I(\theta, \varphi)$ is
minimized) and $\theta = \pi$ (for which it is maximized). The intensity
therefore increases monotonically from the spot center to the antipode,
as stated in \textbf{1}. The value at the minimum is
%
\begin{align}
    \begin{split}
        I_{\mathrm{min}} & = A - \dfrac{B}{\sqrt{C - 1}} \\
        & = 1 - \delta
        \quad,
    \end{split}
\end{align}
%
as stated in \textbf{2}, and the value at the maximum is
%
\begin{align}
    \begin{split}
        I_{\mathrm{max}} & = A - \dfrac{B}{\sqrt{C + 1}} \\
        & = 1
        \quad,
    \end{split}
\end{align}
%
as stated in \textbf{3}.
Finally, to show \textbf{4}, let us compute the half width at half minimum
$\Delta\theta$ of the intensity profile:
%
\begin{align}
    1 - \left(A - \dfrac{B}{\sqrt{C - \cos{\Delta\theta}}}\right) =
    \dfrac{1}{2}\delta
\end{align}
%
Solving for $\Delta\theta$ yields
%
\begin{align}
    \Delta\theta =
    \cos^{-1} \left[ \dfrac{2 + 3 c r (2 + c r)}{2 (1 + c r)^3} \right]
    \quad.
\end{align}
%
Differentiation with respect to $r$ yields
%
\begin{align}
    \dfrac{\mathrm{d}\Delta\theta}{\mathrm{d}r} =
    \frac{3 c^2 r \left(2 + c r\right)}{\left(1 + c r\right)^4
        \sqrt{\frac{c^2 r^2 \left(2 + c r\right)^2 \left(1 + 2 c r\right)
                \left(3 + 2 c r\right)}{\left(1 + c r\right)^6}}}
    \quad,
\end{align}
%
which is positive definite for all $c, r > 0$.

\subsection{Distributions over $r$ and $\delta$}
%
We model the normalized spot radius $r$ as a random variable drawn from
a Beta distribution, whose PDF is given by
%
\begin{align}
    p \big(r \, \big| \, \alpha, \beta \big)
     & =
    \dfrac{\Gamma(\alpha + \beta)}{\Gamma(\alpha)\Gamma(\beta)}
    r^{\alpha - 1}
    (1 - r)^{\beta - 1}
    \quad,
\end{align}
%
where $\Gamma$ is the Gamma function and $\alpha$ and $\beta$ are
hyperparameters characterizing the shape of the distribution.
%
Since we require the brghtness $b = 1 - \delta$ at the center of the spot to
be non-negative, it is convenient to parametrize the distribution of $b$
instead of $\delta$. We model $b$ as a random variable drawn from a log-normal
distribution:
%
\begin{align}
    p \big(b \, \big| \, \mu, \nu \big)
     & =
    \frac{1}{b\sqrt{2\pi\nu}}
    \exp\left[
        -\dfrac{\left(\ln b - \mu\right)^2}{2\nu}
        \right]
    \quad,
\end{align}
%
where $\mu$ and $\nu$ are the mean and variance of the log-normal, respectively.
The joint distribution over $r$ and $b$ is therefore separable,
by construction:
%
\begin{align}
    p \big( r, b \, \big| \, \alpha, \beta, \mu, \nu \big)
     & =
    p \big(r \, \big| \, \alpha, \beta \big)
    \,
    p \big(b \, \big| \, \mu, \nu \big)
    \quad.
\end{align}
%

\subsection{First moment}
%
The first moment of the distribution of $y_{lm}$ over $r$ and $b$ is
%
\begin{align}
    \label{eq:Erb}
    \begin{split}
        \mathrm{E} \big[ y_{lm} \big] &=
        \int\limits_0^\infty \int\limits_0^1
        y_{lm} (r, 1 - b)
        p \big(r, 1 - b \, \big| \, \alpha, \beta, \mu, \nu \big)
        \mathrm{d} r
        \,
        \mathrm{d} b
        \\
        &=
        \int\limits_0^\infty
        \mathrm{E}_r \big[ y_{lm} \big]
        p \big(b \, \big| \, \mu, \nu \big)
        \mathrm{d} b
        \quad,
    \end{split}
    %
    \\
    \intertext{where}
    \label{eq:Er}
    %
    \mathrm{E}_r \big[ y_{lm} \big]
     & =
    \int\limits_0^1
    y_{lm} (r, 1 - b)
    p \big(r \, \big| \, \alpha, \beta \big)
    \mathrm{d} r
    \quad.
\end{align}
%
Expanding Equation~(\ref{eq:Er}), we obtain
%
\begin{align}
    \mathrm{E}_r \big[ y_{lm} \big]
     & =
    \begin{dcases}
        \dfrac{\Gamma(\alpha + \beta)}{\Gamma(\alpha)\Gamma(\beta)}
        \int\limits_0^1
        \left(
        1 - \dfrac{(1 - b) c r}{2 (1 + c r)}
        \right)
        r^{\alpha - 1}
        (1 - r)^{\beta - 1}
        \mathrm{d} r
         &
        \qquad
        l = m = 0    \\[2em]
        -\dfrac{\Gamma(\alpha + \beta)}{\Gamma(\alpha)\Gamma(\beta)}
        \int\limits_0^1
        \dfrac{(1 - b) c r \left( 2 + c r \right)r^{\alpha - 1}
        (1 - r)^{\beta - 1}}
        {2 \sqrt{2l + 1} (1 + c r)^{l + 1}}
        \mathrm{d} r
         &
        \qquad
        l > 0, m = 0 \\[2em]
        0
         &
        \qquad m \ne 0
        \quad.
    \end{dcases}
\end{align}
%
Thanks to the dependence of the spherical harmonic coefficients on only
powers of $r$ and $(1 + cr)$, the integrals above may be expressed in closed
form in terms of the hypergeometric function ${_2F_1}$:
%
\begin{align}
    \mathrm{E}_r \big[ y_{lm} \big]
     & =
    \begin{dcases}
        1 -
        \dfrac{(1 - b) \alpha c}{2(\alpha + \beta)} F_l
         &
        \qquad
        l = m = 0    \\[2em]
        -\dfrac{(1 - b) \alpha c}{\sqrt{2l + 1} (\alpha + \beta)}
        \left[
            F_l + \dfrac{(\alpha + 1)c}{2(\alpha + \beta + 1)} G_l
            \right]
         &
        \qquad
        l > 0, m = 0 \\[2em]
        0
         &
        \qquad m \ne 0
    \end{dcases}
\end{align}
%
where we define
%
\begin{align}
    \begin{split}
        F_l & \equiv {_2F_1}(l + 1, \alpha + 1; \alpha + \beta + 1; -c)
        \\[1em]
        G_l & \equiv {_2F_1}(l + 1, \alpha + 2; \alpha + \beta + 2; -c)
        \quad.
    \end{split}
\end{align}
%
Note that for a given value of $\alpha$, $\beta$, and $c$, we need only compute
$F_0$, $F_1$, $G_0$, and $G_1$ directly, since the remaining terms may be
obtained recursively:
%
\begin{align}
    \begin{split}
        F_l &=
        \dfrac{\alpha + \beta + 1 - l}{2l}F_{l - 2}
        - \dfrac{2\alpha + \beta + 2 - 3l}{2l}F_{l - 1}
        \\[1em]
        G_l &=
        \dfrac{\alpha + \beta + 2 - l}{2l}G_{l - 2}
        - \dfrac{2\alpha + \beta + 4 - 3l}{2l}G_{l - 1}
        \quad.
    \end{split}
\end{align}
%
We may now evaluate the integral in Equation~(\ref{eq:Erb}):
%
\begin{align}
    \mathrm{E} \big[ y_{lm} \big] & =
    \begin{dcases}
        \frac{1}{b\sqrt{2\pi\nu}}
        \int\limits_0^\infty
        \left(
        1 -
        \dfrac{(1 - b) \alpha c}{2(\alpha + \beta)} F_l
        \right)
        \\
        \qquad\qquad\times
        \exp\left[
            -\dfrac{\left(\ln b - \mu\right)^2}{2\nu}
            \right]
        \mathrm{d} b
         &
        \qquad\qquad
        l = m = 0
        \\
        -
        \frac{1}{b\sqrt{2\pi\nu}}
        \int\limits_0^\infty
        \dfrac{(1 - b) \alpha c}{\sqrt{2l + 1} (\alpha + \beta)}
        \\
        \qquad\qquad\times
        \left[
            F_l + \dfrac{(\alpha + 1)c}{2(\alpha + \beta + 1)} G_l
            \right]
        \\
        \qquad\qquad\times
        \exp\left[
            -\dfrac{\left(\ln b - \mu\right)^2}{2\nu}
            \right]
        \mathrm{d} b
         &
        \qquad\qquad
        l > 0, m = 0 \\[2em]
        0
         &
        \qquad\qquad m \ne 0
    \end{dcases}
\end{align}
%
Most of the terms in the integrands above are constants, so these
integrals may also be expressed in closed form:
%
\begin{align}
    \mathrm{E} \big[ y_{lm} \big]
     & =
    \begin{dcases}
        1 -
        \dfrac{\gamma \alpha c}{2(\alpha + \beta)} F_l
         &
        \qquad
        l = m = 0    \\[2em]
        -\dfrac{\gamma \alpha c}{\sqrt{2l + 1} (\alpha + \beta)}
        \left[
            F_l + \dfrac{(\alpha + 1)c}{2(\alpha + \beta + 1)} G_l
            \right]
         &
        \qquad
        l > 0, m = 0 \\[2em]
        0
         &
        \qquad m \ne 0
    \end{dcases}
\end{align}
%
where we define
%
\begin{align}
    \gamma \equiv 1 - \exp\left[ \mu + \frac{1}{2}\nu\right]
    \quad.
\end{align}
%

\subsection{Second moment}
%
The second moment of the distribution of $y_{lm}$ over $r$ and $b$ is
%
\begin{align}
    \label{eq:Erb2}
    \begin{split}
        \mathrm{E} \big[ y_{lm} y_{l'm'} \big] &=
        \int\limits_0^\infty \int\limits_0^1
        y_{lm} (r, 1 - b)
        y_{l'm'} (r, 1 - b)
        p \big(r, 1 - b \, \big| \, \alpha, \beta, \mu, \nu \big)
        \mathrm{d} r
        \,
        \mathrm{d} b
        \\
        &=
        \int\limits_0^\infty
        \mathrm{E}_r \big[ y_{lm} y_{l'm'} \big]
        p \big(b \, \big| \, \mu, \nu \big)
        \mathrm{d} b
        \quad,
    \end{split}
    %
    \\
    \intertext{where}
    \label{eq:Er2}
    %
    \mathrm{E}_r \big[ y_{lm} y_{l'm'} \big]
     & =
    \int\limits_0^1
    y_{lm} (r, 1 - b)
    y_{l'm'} (r, 1 - b)
    p \big(r \, \big| \, \alpha, \beta \big)
    \mathrm{d} r
    \quad.
\end{align}
%
Expanding Equation~(\ref{eq:Er2}), we obtain
%
\begin{align}
    \mathrm{E}_r \big[ y_{lm} y_{l'm'} \big]
     & =
    \begin{dcases}
        \dfrac{\Gamma(\alpha + \beta)}{\Gamma(\alpha)\Gamma(\beta)}
        \int\limits_0^1
        \left(
        1 - \dfrac{(1 - b) c r}{2 (1 + c r)}
        \right)^2
        r^{\alpha - 1}
        (1 - r)^{\beta - 1}
        \mathrm{d} r
         &
        \quad\quad
        \parbox[t]{\linewidth}{$l = l',              \\m = m' = 0$}
        \\[2em]
        %
        -\dfrac{\Gamma(\alpha + \beta)}{\Gamma(\alpha)\Gamma(\beta)}
        \int\limits_0^1
        \left(1 - \dfrac{(1 - b) c r}{2 (1 + c r)}\right)
        \\
        \qquad\qquad\quad\quad\times
        \dfrac{(1 - b) c r \left( 2 + c r \right) r^{\alpha - 1}
            (1 - r)^{\beta - 1}}
        {2 \sqrt{2l + 1} (1 + c r)^{l + 1}}
        \mathrm{d} r
         &
        \quad\quad
        \parbox[t]{\linewidth}{\vspace{-3em}$l > 0,  \\l' = 0, \\m = m' = 0$}
        \\[2em]
        %
        -\dfrac{\Gamma(\alpha + \beta)}{\Gamma(\alpha)\Gamma(\beta)}
        \int\limits_0^1
        \left(1 - \dfrac{(1 - b) c r}{2 (1 + c r)}\right)
        \\
        \qquad\qquad\quad\quad\times
        \dfrac{(1 - b) c r \left( 2 + c r \right) r^{\alpha - 1}
            (1 - r)^{\beta - 1}}
        {2 \sqrt{2l' + 1} (1 + c r)^{l' + 1}}
        \mathrm{d} r
         &
        \quad\quad
        \parbox[t]{\linewidth}{\vspace{-3em}$l = 0,  \\l' > 0, \\m = m' = 0$}
        \\[2em]
        %
        -\dfrac{\Gamma(\alpha + \beta)}{\Gamma(\alpha)\Gamma(\beta)}
        \int\limits_0^1
        \dfrac{(1 - b)^2 c^2 r^2 \left( 2 + c r \right)^2 r^{\alpha - 1}
            (1 - r)^{\beta - 1}}
        {4 \sqrt{(2l + 1)(2l' + 1)} (1 + c r)^{l + l' + 2}}
        \mathrm{d} r
         &
        \quad\quad
        \parbox[t]{\linewidth}{\vspace{-1.5em}$l > 0 \\l' > 0 \\m = m' = 0$}
        \\[2em]
        %
        0
         &
        \quad\quad m, m' \ne 0
        \quad.
    \end{dcases}
\end{align}

%\bibliography

\end{document}
