% !TeX root = ./ms.tex
\documentclass[modern]{aastex62}

% Load the corTeX style definitions
% !TeX root = ./ms.tex
% All the packages
\usepackage{url}
\usepackage{amsmath}
\usepackage{mathtools}
\usepackage{amssymb}
\usepackage{natbib}
\usepackage{graphicx}
\usepackage{calc}
\usepackage{etoolbox}
\usepackage{xspace}
\usepackage[T1]{fontenc} % https://tex.stackexchange.com/a/166791
\usepackage{textcomp}
\usepackage{ifxetex}
\ifxetex
  \usepackage{fontspec}
  \defaultfontfeatures{Extension = .otf}
\fi
\usepackage{fontawesome}
\usepackage{listings}
\usepackage{nicefrac}
%\usepackage{bm}
\usepackage{booktabs}
\usepackage{longtable}

% Shorthand for this paper
\newcommand{\starry}{\textsf{starry}\xspace}
\newcommand{\Python}{\textsf{Python}\xspace}
\newcommand{\xxx}[1]{{\color{red}#1}}
\newcommand{\quadquad}{\quad\quad\quad\quad}

% References to text content
\newcommand{\documentname}{\textsl{article}}
\newcommand{\figureref}[1]{\ref{fig:#1}}
\newcommand{\Figure}[1]{Figure~\figureref{#1}}
\newcommand{\figurelabel}[1]{\label{fig:#1}}
\renewcommand{\eqref}[1]{\ref{eq:#1}}
\newcommand{\Eq}[1]{Equation~(\eqref{#1})}
\newcommand{\eq}[1]{\Eq{#1}}
\newcommand{\eqalt}[1]{Equation~\eqref{#1}}

% Add code, proof, and animation hyperlinks
\definecolor{linkcolor}{rgb}{0.1216,0.4667,0.7059}
\definecolor{testpasscolor}{rgb}{0.13333333,0.5254902,0.22745098}
\definecolor{testfailcolor}{rgb}{0.79607843,0.14117647,0.19215686}
\newcommand{\codeicon}{{\color{linkcolor}\faFileCodeO}}
\newcommand{\prooficon}{{\color{linkcolor}\faPencilSquareO}}
\newcommand{\testpassicon}{{\color{testpasscolor}\faCheckCircle}}
\newcommand{\testfailicon}{{\color{testfailcolor}\faTimesCircle}}
\newcommand{\codelink}[1]{\href{https://github.com/rodluger/starry_process/blob/0f5cd85041405565c68a93eca39244838420c99d/tex/figures/#1.py}{\codeicon}\,\,}
\newcommand{\animlink}[1]{\href{https://github.com/rodluger/starry_process/blob/0f5cd85041405565c68a93eca39244838420c99d/tex/figures/#1.gif}{\animicon}\,\,}
\newcommand{\prooflink}[1]{\href{https://github.com/rodluger/starry_process/blob/0f5cd85041405565c68a93eca39244838420c99d/tex/tests/#1.py}{\raisebox{-0.1em}{\input{tests/#1.tex}}}}
\newcommand{\cilink}[1]{\href{https://dev.azure.com/rodluger/starry_process/_build}{#1}}


% Define a proof environment for open source equation proofs
\newtagform{eqtag}[]{(}{)}
\newcommand{\currentlabel}{None}
\newenvironment{proof}[1]{%
  \ifstrempty{#1}{%
    \renewtagform{eqtag}[]{\raisebox{-0.1em}{{\color{red}\faPencilSquareO}}\,(}{)}%
  }{%
    \renewtagform{eqtag}[]{\prooflink{#1}\,(}{)}%
  }%
  \usetagform{eqtag}%
  \renewcommand{\currentlabel}{#1}
  \align%
}{%
  \endalign%
  \renewtagform{eqtag}[]{(}{)}%
  \usetagform{eqtag}%
  \message{<<<\currentlabel: \theequation>>>}%
}

% Define the `oscaption` command for open source figure captions
\newcommand{\oscaption}[2]{\caption{#2 \codelink{#1}}}

% Code examples
\definecolor{codegreen}{rgb}{0,0.6,0}
\definecolor{codegray}{rgb}{0.5,0.5,0.5}
\definecolor{codepurple}{rgb}{0.58,0,0.82}
\definecolor{backcolour}{rgb}{0.95,0.95,0.95}
\lstdefinestyle{mystyle}{
  backgroundcolor=\color{backcolour},
  commentstyle=\color{codegreen},
  keywordstyle=\color{magenta},
  numberstyle=\tiny\color{codegray},
  stringstyle=\color{codepurple},
  basicstyle=\small\ttfamily,
  breakatwhitespace=false,
  breaklines=true,
  captionpos=b,
  keepspaces=true,
  numbers=left,
  numbersep=5pt,
  showspaces=false,
  showstringspaces=false,
  showtabs=false,
  tabsize=2,
  aboveskip=1em,
  belowskip=1em,
  keywords=[2]{map},
  keywordstyle=[2]{\color{black!80!black}},
  upquote=true
}
\lstset{style=mystyle}

% Typography obsessions
\setlength{\parindent}{3.0ex}
\renewcommand\quad{\hskip\fontdimen3\font}

% https://tex.stackexchange.com/a/184474
\usepackage{stackengine,scalerel}
\def\lnlam{\ThisStyle{\ensurestackMath{\stackon[-2.4\LMpt]{%
        \SavedStyle\lambda}{\kern-.5pt\kern\LMpt\rule{1\LMex}{.25pt+.15\LMpt}}}}}

% Load custom style
% Packages
\usepackage{xifthen}
\usepackage{stackengine}
\usepackage{tabstackengine}
\usepackage{array}
\usepackage{upgreek}
\usepackage[bbgreekl]{mathbbol}
\usepackage{afterpage}
\usepackage[bb=boondox]{mathalpha}

% Misc. macros
\newcommand{\LMAX}{15\xspace}

% Integrals
\newcommand{\dd}{\ensuremath{\text{d}}}

% Special functions
\newcommand{\sgn}{{\text{sgn}}}
\newcommand{\atantwo}{{\text{arctan2}}}

% Cartesian unit vectors
\newcommand{\xhat}{\ensuremath{\pmb{\hat{x}}}\xspace}
\newcommand{\yhat}{\ensuremath{\pmb{\hat{y}}}\xspace}
\newcommand{\zhat}{\ensuremath{\pmb{\hat{z}}}\xspace}

% Other
\DeclarePairedDelimiter\ceil{\lceil}{\rceil}
\DeclarePairedDelimiter\floor{\lfloor}{\rfloor}

% Inverse diagonal dots
\makeatletter
\def\Ddots{\mathinner{\mkern1mu\raise\p@
                \vbox{\kern7\p@\hbox{.}}\mkern2mu
                \raise4\p@\hbox{.}\mkern2mu\raise7\p@\hbox{.}\mkern1mu}}
\makeatother

% Imaginary unit
\DeclareFontFamily{U}{mathc}{}
\DeclareFontShape{U}{mathc}{m}{it}{<->s*[1.03] mathc10}{}
\DeclareMathAlphabet{\mathscr}{U}{mathc}{m}{it}
\DeclareMathOperator{\imag}{\mathscr{i}}

% Bibliography
\bibliographystyle{aasjournal}

\usepackage{etoolbox}
\makeatletter % we need to patch \env@cases that has @ in its name
\patchcmd{\env@cases}{\quad}{\qquad\qquad}{}{}
\makeatother

\usepackage{enumitem}

% Begin!
\begin{document}

% Title
\title{%
    \textbf{
        Interpretable Gaussian Processes for Stellar Light Curves
    }
}

% Author list
\author[0000-0002-0296-3826]{Rodrigo Luger}\altaffiliation{Flatiron Fellow}
\email{rluger@flatironinstitute.org}
\affil{Center~for~Computational~Astrophysics, Flatiron~Institute, New~York, NY}
\affil{Virtual~Planetary~Laboratory, University~of~Washington, Seattle, WA}
%

\keywords{methods: analytic}

% \begin{abstract}
%     Abstract here.
%     %
%     \href{https://github.com/rodluger/starry_process}{\color{linkcolor}\faGithub}
% \end{abstract}

\section{Introduction}
\label{sec:intro}
\xxx{Talk about starry, gps, and whatnot.}

\section{Spherical Harmonics}
\label{sec:ylm}
%
\xxx{Introduce the spherical harmonics.}

The real spherical harmonics are indexed by their degree $l \in [0, \infty]$
and order $m \in [-l, l]$. It is convenient to collect the spherical harmonic coefficients of
a given expansion into a vector $\mathbf{y}$ indexed by a single
integer $n$, where
%
\begin{align}
    \label{eq:n}
    n = l^2 + l + m
\end{align}
%
and, conversely,
%
\begin{align}
    \label{eq:lm}
    \begin{split}
        l & = \floor{\sqrt{n}}
        \\
        m & = n - l^2 - l
        \quad.
    \end{split}
\end{align}

\section{Gaussian Process}
\label{sec:gp}
%

% The vector $\pmb{\theta}$ includes physical properties of the star such
% as its inclination $i$ and rotational period $P$ as well as parameters
% describing the shape of the probability density function (PDF) governing
% the distribution of features on the surface.

Let
$\mathbf{f} = \left( f_0 \, f_1 \, \cdots \,  f_K \right)^\top$
denote a vector of $K$ flux measurements at times
$\left( t_0 \,  t_1 \,  \cdots \, t_K \right)^\top$.
Conditioned on certain physical properties of the star, $\pmb{\theta}$,
we wish to compute the mean $\pmb{\mu}(\pmb{\theta})$ and
covariance $\pmb{\Sigma}(\pmb{\theta})$
of $\mathbf{f}$, which together fully specify our GP in flux.
%
As with any random variable, the mean and covariance may be computed from
the expectation value of $\mathbf{f}$ and
$\mathbf{f}\,\mathbf{f}^\top$, respectively:
%
\begin{align}
    \label{eq:mean}
    \pmb{\mu}(\pmb{\theta})
     & = \mathrm{E} \Big[ \mathbf{f} \, \Big| \, \pmb{\theta} \Big]
    \\
    \label{eq:cov}
    \pmb{\Sigma}(\pmb{\theta})
     & = \mathrm{E} \Big[ \mathbf{f} \, \mathbf{f}^\top \, \Big| \, \pmb{\theta} \Big] - \pmb{\mu}^2(\pmb{\theta})
\end{align}
%
where the squaring operation in Equation~(\ref{eq:cov}) is
performed element-wise.
%
In \citet{Luger2019} we showed that $\mathbf{f}$ may be computed from a
linear operation on the vector of spherical harmonic coefficients
describing the surface, $\mathbf{y}$:
%
\begin{align}
    \label{eq:fAy}
    \mathbf{f} = \mathbf{A} \, \mathbf{y}
    \quad,
\end{align}
%
where $\mathbf{A}$ is the \starry design matrix, which is implicitly
a function of $\pmb{\theta}$ (as it depends on the stellar inclination
and rotation period, for example).
%
Given Equation~(\ref{eq:fAy}),
we may write the mean and covariance of our flux GP as
%
\begin{align}
    \pmb{\mu}(\pmb{\theta})
     & = \mathbf{A}(\pmb{\theta}) \, \pmb{\mu}_{\mathbf{y}}(\pmb{\theta})
    \\
    \pmb{\Sigma}(\pmb{\theta})
     & = \mathbf{A}(\pmb{\theta}) \, \pmb{\Sigma}_{\mathbf{y}} \, \mathbf{A}^\top(\pmb{\theta})
    \quad,
\end{align}
%
where
%
\begin{align}
    \label{eq:mean_y}
    \pmb{\mu}_{\mathbf{y}}(\pmb{\theta})
     & = \mathrm{E} \Big[ \mathbf{y} \, \Big| \, \pmb{\theta} \Big]
    \\
    \label{eq:cov_y}
    \pmb{\Sigma}_{\mathbf{y}}(\pmb{\theta})
     & = \mathrm{E} \Big[ \mathbf{y} \, \mathbf{y}^\top \, \Big| \, \pmb{\theta} \Big] - \pmb{\mu}_{\mathbf{y}}^2(\pmb{\theta})
\end{align}
%
are the mean and covariance of the GP in the spherical harmonics basis.
The bulk of the math in this paper is devoted to computing
the expectations in the expressions above, which
are given by the integrals
%
\begin{align}
    \label{eq:exp_y}
    \mathrm{E} \Big[ \mathbf{y} \, \Big| \, \pmb{\theta} \Big]
     & =
    \int \mathbf{y}(\mathbf{x} ) \, p(\mathbf{x} \, \big| \, \pmb{\theta})\mathrm{d}\mathbf{x}
    \\
    \label{eq:exp_yy}
    \mathrm{E} \Big[ \mathbf{y} \, \mathbf{y}^\top \, \Big| \, \pmb{\theta} \Big]
     & =
    \int \mathbf{y}(\mathbf{x} ) \mathbf{y}^\top(\mathbf{x} ) \, p(\mathbf{x} \, \big| \, \pmb{\theta})\mathrm{d}\mathbf{x}
    \quad,
\end{align}
%
where $\mathbf{x}$ is a random vector-valued variable corresponding to a particular
distribution of features on the surface  (i.e., the size and location of star spots)
and $p(\mathbf{x} \, \big| \, \pmb{\theta})$ is its probability density
function (PDF).

In the sections that follow we will show that for suitable choices of $\mathbf{y}(\mathbf{x})$
and $p(\mathbf{x} \, \big| \, \pmb{\theta})$, the integrals in the expressions
above have closed form solutions that may be evaluated quickly.
%
As we are specifically interested in modeling the effect of star spots
on stellar light curves, we let
%
\begin{align}
    \mathbf{x} = \left( \lambda \,\, \phi \,\, \xi \,\, r \right)^\top
\end{align}
%
and
%
\begin{align}
    \label{eq:RRs}
    \mathbf{y}(\mathbf{x}) =
    \mathbf{R}_{\hat{\mathbf{y}}}(\lambda)
    \,
    \mathbf{R}_{\hat{\mathbf{x}}}(\phi)
    \,
    \xi
    \,
    \mathbf{s}(r)
    \quad,
\end{align}
%
where $\lambda$ is the longitude of a spot, $\phi$ is its latitude, $\xi$
is its contrast, and $r$ is its radius. The vector function $\mathbf{s}(r)$
returns the spherical harmonic expansion of a negative unit brightness
circular spot at $\lambda = \phi = 0$,
$\mathbf{R}_{\hat{\mathbf{x}}}(\phi)$ is the Wigner matrix that rotates the
expansion about $\hat{\mathbf{x}}$ such that the spot is centered at a
latitude $\phi$, and $\mathbf{R}_{\hat{\mathbf{y}}}(\lambda)$ is the Wigner
matrix that then rotates the
expansion about $\hat{\mathbf{y}}$ such that the spot is centered at a
longitude $\lambda$; these three functions are detailed in the sections below.
%
Equation~(\ref{eq:RRs}) thus provides a way of converting a random variable
$\mathbf{x}$ describing the size, brightness, and position of a spot to the
corresponding representation in terms of spherical harmonics.
%
Note, importantly, that we are not interested in any specific value of
$\mathbf{y}$; rather, we would like to know its expectation value under
the probability distribution governing the different spot properties $\mathbf{x}$,
i.e., $p(\mathbf{x} \, \big| \, \pmb{\theta})$.
%
For simplicity, we assume that $p(\mathbf{x} \, \big| \, \pmb{\theta})$
is separable in each of the four spot properties:
%
\begin{align}
    p(\mathbf{x} \, \big| \, \pmb{\theta})
    =
    p(\lambda \, \big| \, \pmb{\theta}_{\lambda}) \,
    p(\phi \, \big| \, \pmb{\theta}_{\phi})\,
    p(\xi \, \big| \, \pmb{\theta}_{\xi}) \,
    p(r \, \big| \, \pmb{\theta}_{r})
    \quad,
\end{align}
%
where
%
\begin{align}
    \pmb{\theta} = \left(
    \pmb{\theta}_{\lambda} \, \,
    \pmb{\theta}_{\phi} \, \,
    \pmb{\theta}_{\xi} \, \,
    \pmb{\theta}_{r} \right)^\top
    \quad.
\end{align}
%
This allows us to rewrite the expectation integrals (\ref{eq:exp_y})
and (\ref{eq:exp_yy}) as
%
\begin{align}
    \label{eq:exp_y_sep}
    \mathrm{E} \Big[ \mathbf{y} \, \Big| \, \pmb{\theta} \Big]
     & =
    \mathbf{e_4}(\pmb{\theta})
    \\[1em]
    \label{eq:exp_yy_sep}
    \mathrm{E} \Big[ \mathbf{y} \, \mathbf{y}^\top \, \Big| \, \pmb{\theta} \Big]
     & =
    \mathbf{E_4}(\pmb{\theta})
\end{align}
%
where we define the first moment integrals
%
\begin{align}
    \label{eq:e1}
    \mathbf{e_1}(\pmb{\theta}_r)
     & \equiv
    \int
    \mathbf{s}(r) \,
    p(r \, \big| \, \pmb{\theta}_{r}) \,
    \mathrm{d}r
    %
    \\[1em]
    \label{eq:e2}
    \mathbf{e_2}(\pmb{\theta}_b, \mathbf{e_1})
     & \equiv
    \int
    \xi \,
    \mathbf{e_1} \,
    p(\xi \, \big| \, \pmb{\theta}_{\xi}) \,
    \mathrm{d}\xi
    %
    \\[1em]
    %
    \label{eq:e3}
    \mathbf{e_3}(\pmb{\theta}_\phi, \mathbf{e_2})
     & \equiv
    \int
    \mathbf{R}_{\hat{\mathbf{x}}}(\phi) \,
    \mathbf{e_2} \,
    p(\phi \, \big| \, \pmb{\theta}_{\phi}) \,
    \mathrm{d}\phi
    %
    \\[1em]
    %
    \label{eq:e4}
    \mathbf{e_4}(\pmb{\theta}_\lambda, \mathbf{e_3})
     & \equiv
    \int
    \mathbf{R}_{\hat{\mathbf{y}}}(\lambda) \,
    \mathbf{e_3} \,
    p(\lambda \, \big| \, \pmb{\theta}_{\lambda}) \,
    \mathrm{d}\lambda
\end{align}
%
and the second moment integrals
%
\begin{align}
    \label{eq:E1}
    \mathbf{E_1}(\pmb{\theta}_r)
     & \equiv
    \int
    \mathbf{s}(r) \, \mathbf{s}^\top(r) \,
    p(r \, \big| \, \pmb{\theta}_{r}) \,
    \mathrm{d}r
    %
    \\[1em]
    %
    \label{eq:E2}
    \mathbf{E_2}(\pmb{\theta}_b, \mathbf{E_1})
     & \equiv
    \int
    \xi^2 \,
    \mathbf{E_1} \,
    \mathbf{E_1}^\top \,
    p(\xi \, \big| \, \pmb{\theta}_b)
    \mathrm{d}\xi
    %
    \\[1em]
    %
    \label{eq:E3}
    \mathbf{E_3}(\pmb{\theta}_\phi, \mathbf{E_2})
     & \equiv
    \int
    \mathbf{R}_{\hat{\mathbf{x}}}(\phi) \,
    \mathbf{E_2} \,
    \mathbf{E_2}^\top \,
    \mathbf{R}_{\hat{\mathbf{x}}}^\top(\phi) \,
    p(\phi \, \big| \, \pmb{\theta}_{\phi})
    \mathrm{d}\phi
    %
    \\[1em]
    %
    \label{eq:E4}
    \mathbf{E_4}(\pmb{\theta}_\lambda, \mathbf{E_3})
     & \equiv
    \int
    \mathbf{R}_{\hat{\mathbf{y}}}(\lambda) \,
    \mathbf{E_3} \,
    \mathbf{E_3}^\top \,
    \mathbf{R}_{\hat{\mathbf{y}}}^\top(\lambda) \,
    p(\lambda \, \big| \, \pmb{\theta}_{\lambda})
    \mathrm{d}\phi
    \quad.
\end{align}
%
We devote the next four sections to the computation of these eight
integrals.

%
\section{The Size Integrals: \lowercase{$\mathbf{e_1}$} and $\mathbf{E_1}$}
\label{sec:size}
%
\subsection{Functional form}
\label{sec:size-function}
%
We adopt the following expression for the $n^{\mathrm{th}}$ term in $\mathbf{s}$
in the expansion of a dark, unit contrast circular spot at
$\lambda = \phi = 0$:
%
\begin{align}
    \label{eq:s}
    s_{n}(r) =
    \begin{cases}
        \dfrac{r' \delta_{l0}}{2}
        -\dfrac{r' \left( 2 + r' \right)}
        {2 \sqrt{2l + 1} (1 + r')^{l + 1}}
         & m = 0
        \\
        0
         & \mathrm{otherwise}
    \end{cases}
\end{align}
%
where
$l = l(n), m = m(n)$ (Equation~\ref{eq:lm}) and
$\delta$ is the Kronecker delta. The parameter $r'$ is defined as
%
\begin{align}
    \label{eq:rprime}
    r' \equiv c_0 + c_1 r
    \quad,
\end{align}
%
where $c_0, c_1 > 0$ are scaling constants (defined in more detail below)
and $r \in [0, 1]$ is the normalized spot radius.
The expression in Equation~(\ref{eq:s}) is convenient because it satisfies
five important properties:
%
\begin{enumerate}[itemsep=2pt,parsep=1pt,label=\textbf{\arabic*}]
    \item The surface intensity is azimuthally symmetric about the spot center
    \item The surface intensity monotonically increases away from the spot center
    \item The surface intensity at the spot center is $-1$
    \item The surface intensity at the antipode of the spot center is zero
    \item The size of the spot increases monotonically with $r$
\end{enumerate}
%
These properties may be demonstrated by considering the
expression for the surface intensity at a given polar angle $\theta$ and
azimuth $\varphi$:
%
\begin{align}
    \begin{split}
        I(\theta, \varphi)
        & =
        \sum\limits_{n=0}^\infty
        s_{n} Y_{n}(\theta, \varphi) \\
        & =
        \sum\limits_{n=0}^\infty
        s_{n} \sqrt{2l + 1} \delta_{m0} P_l(\cos\theta)
    \end{split}
\end{align}
%
where $Y_n$ is the spherical harmonic of degree $l(n)$ and order $m(n)$
(see Equation~\ref{eq:lm}),
$P_l$ is the Legendre polynomial of degree $l$,
and we have implicitly
assumed a normalization such that the integral of our expansion over
the unit sphere is $4\pi$.
Note that the expression above is independent of the azimuth $\varphi$,
a consequence of the fact that all harmonics with $m \ne 0$ are zero; this
expansion is therefore azimuthally symmetric about the spot center, as
stated in \textbf{1}.
%
Combining the above expression with Equation~(\ref{eq:s}) and
rearranging, we may write
%
\begin{align}
    \label{eq:Igen}
    I(\theta, \varphi) =
    I(\theta) =
    \dfrac{r'}{2}
    -
    \dfrac{r' \left( 2 + r' \right)}{2 (1 + r')}
    \sum\limits_{l=0}^\infty \left(\dfrac{1}{1 + r'}\right)^l P_l(\cos\theta)
    \quad.
\end{align}
%
The summation in Equation~(\ref{eq:Igen}) has a closed-form expression in
terms of the generating function of the Legendre polynomials:
%
\begin{align}
    \label{eq:gen}
    \sum\limits_{l=0}^\infty t^l P_l(\cos\theta) = \frac{1}{\sqrt{1 + t^2 - 2 t \cos\theta}}
    \quad,
\end{align}
%
so we may express the intensity as a function of polar angle
in the fairly simple form
%
\begin{align}
    \label{eq:Ifinal}
    I(\theta) & = A - \frac{B}{\sqrt{C - \cos\theta}}
    \quad,
    %
    \\
    \intertext{where}
    %
    \begin{split}
        A & = \dfrac{r'}{2}                      \\
        B & = r' (2 + r') \sqrt{\dfrac{1}{8 + 8 r'}} \\
        C & = \dfrac{1 + (1 + r')^2}{2 + 2 r'}
    \end{split}
\end{align}
%
are positive constants.

Differentiating Equation~(\ref{eq:Ifinal}) with respect to $\theta$, we have
%
\begin{align}
    \label{eq:Ideriv}
    \dfrac{\mathrm{d}I(\theta)}{\mathrm{d}\theta} & =
    -\frac{B\sin\theta}{2(C - \cos\theta)^\frac{3}{2}}
    \quad,
\end{align}
%
which is zero only for $\theta = 0$ (for which $I(\theta)$ is
minimized) and $\theta = \pi$ (for which it is maximized). The intensity
therefore increases monotonically from the spot center to the antipode,
as stated in \textbf{2}. The value at the minimum is
%
\begin{align}
    \begin{split}
        I_{\mathrm{min}} & = A - \dfrac{B}{\sqrt{C - 1}} \\
        & = -1
        \quad,
    \end{split}
\end{align}
%
as stated in \textbf{3}, and the value at the maximum is
%
\begin{align}
    \begin{split}
        I_{\mathrm{max}} & = A - \dfrac{B}{\sqrt{C + 1}} \\
        & = 0
        \quad,
    \end{split}
\end{align}
%
as stated in \textbf{4}.
Finally, to show \textbf{5}, let us compute the half width at half minimum
$\Delta\theta$ of the intensity profile:
%
\begin{align}
    A - \dfrac{B}{\sqrt{C - \cos{\Delta\theta}}} =
    -\dfrac{1}{2}
\end{align}
%
Solving for $\Delta\theta$ yields
%
\begin{align}
    \label{eq:hwhm}
    \Delta\theta =
    \cos^{-1} \left[ \dfrac{2 + 3 r' (2 + r')}{2 (1 + r')^3} \right]
    \quad.
\end{align}
%
Differentiation with respect to $r'$ yields
%
\begin{align}
    \dfrac{\mathrm{d}\Delta\theta}{\mathrm{d}r'} =
    \frac{3}{\left(1 + r'\right)
        \sqrt{\left(1 + 2 r'\right)
            \left(3 + 2 r'\right)}}
    \quad,
\end{align}
%
which is positive definite for all $r' > 0$.

\begin{figure}[p!]
    \begin{centering}
        \includegraphics[width=\linewidth]{figures/hwhm.pdf}
        \oscaption{hwhm}{%
            \emph{Left.} The width $\Delta\theta$
            (Equation~\ref{eq:hwhm}) of the
            spot as a function of the radius parameter $r'$ (bottom axis)
            and the normalized radius $r$ (top axis), computed for
            $l_{\mathrm{max}} = 20$. The green dots indicate the
            minimum value of $r'$ for which the fractional error in the
            spot intensity due
            to the truncated expansion is less than one percent
            and the value of $r'$ corresponding to
            $\Delta\theta = 75^\circ$, respectively. These points correspond to
            the minimum and maximum spot sizes we consider at
            $l_{\mathrm{max}} = 20$
            (blue shading). Note that
            in this regime, the width of the spot scales approximately
            logarithmically with $r'$.
            %
            \emph{Right.} The spot width $\Delta\theta$ below which the
            fractional error in the intensity exceeds various thresholds
            as a function of maximum spherical harmonic degree
            $l_{\mathrm{max}}$. The value of $\Delta\theta$ for a tolerance
            of one percent at $l_{\mathrm{max}} = 20$ is indicated as the
            green dot for reference.
            \label{fig:hwhm}
        }
    \end{centering}
\end{figure}

\begin{figure}[p!]
    \begin{centering}
        \includegraphics[width=\linewidth]{figures/spot_expansion.pdf}
        \oscaption{spot_expansion}{%
            Polar intensity profiles for spots with different normalized
            radii $r$ in the range $(0, 1]$, computed at spherical harmonic
            degree $l_{\mathrm{max}} = 20$. Dotted black curves show profiles
            corresponding to $r < 0$ (corresponding to $r' < c_0$
            in Equation~\ref{eq:rprime}), for which the expansion does
            not converge, leading to ringing and a smaller spot contrast.
            \label{fig:spot_expansion}
        }
    \end{centering}
\end{figure}

It is important to note that the
properties of the spot expansion described above
(in particular its monotonicity and values at the spot center and antipode)
are inherited from the generating function of the Legendre polynomials
(Equation~\ref{eq:gen}). These properties are therefore only rigorously
true when the expansion is taken to spherical
harmonic degree $l_{\mathrm{max}} = \infty$. For truncated expansions,
a value of $r'$ in Equation~(\ref{eq:s}) that is too small will lead
to both ringing and a value at the spot center that is in general larger
than $-1$.

To circumvent these issues, we impose a minimum radius
$r'_{\mathrm{min}}$, whose value is
a function of the degree of the expansion. Since the normalized spot radius
$r$ is defined in the range $[0, 1]$ (Equation~\ref{eq:rprime}), this
minimum value is equal to $c_0$. We compute $c_0 = c_0(l_{\mathrm{max}})$
by numerically finding the
value of $r'$ below which the fractional error in the intensity at the
center of the spot exceeds some threshold, which we take to be 0.01.
Since the half width of the spot $\Delta\theta$ approaches $\nicefrac{\pi}{2}$ as
$r' \rightarrow \infty$ (Equation~\ref{eq:hwhm}), we also impose a
maximum radius $r'_{\mathrm{max}}$, which we define to be the radius
corresponding to
$\Delta\theta = \nicefrac{5\pi}{12} = 75^\circ$. This choice is rather
arbitrary, but motivated by the fact that below this value, $\Delta\theta$
scales approximately logarithmically with $r'$.

The left panel of Figure~\ref{fig:hwhm} shows this scaling. The minimum
and maximum values of $r'$ for $l_{\mathrm{max}} = 20$ are indicated
for reference; these correspond to $r = 0$ and $r = 1$, respectively
(top axis). The right panel of the figure shows the minimum spot half width
$\Delta\theta$ as a function of the spherical harmonic degree of the
expansion for different error thresholds: ten percent (dashed), one
percent (solid), and one part per thousand (dotted).
For $l_{\mathrm{max}} = 20$, the minimum spot half width is about 17$^\circ$.
To model spots smaller than this, one must go to larger $l_{\mathrm{max}}$;
note, however, how slowly $\Delta\theta$ decreases above
$l_{\mathrm{max}} = 20$.

The intensity as a function of polar angle away from the spot center
is shown in Figure~\ref{fig:spot_expansion} for a range of $r$ values.
Three spots with $r < 0$ are also shown; note the presence of
ringing and the fact that the intensity at the spot center is significantly
larger than $-1$.

\subsection{PDF}
\label{sec:size-pdf}
%
Our goal now is to characterize the distribution of $\mathbf{s}$ as a
function of parameters $\pmb{\theta}_r$ describing the distribution of
spot radii. Since $r \in [0, 1]$, we choose to model $r$ as a random
Beta-distributed variable; as we will see, this choice will allow us to
analytically compute the first two moments of the distribution of
$\mathbf{s}$ conditioned on $\pmb{\theta}_r$.

%
The Beta distribution in $r$ has hyperparameters
%
\begin{align}
    \pmb{\theta}_r = \left(
    \alpha \, \, \, \,
    \beta \right)^\top
\end{align}
%
and PDF given by
%
\begin{align}
    \label{eq:pdf_r}
    p \big(r \, \big| \, \pmb{\theta}_r \big)
     & =
    \dfrac{\Gamma(\alpha + \beta)}{\Gamma(\alpha)\Gamma(\beta)}
    r^{\alpha - 1}
    (1 - r)^{\beta - 1}
    \quad,
\end{align}
%
with support in $0 \leq r \leq 1$,
where $\Gamma$ is the Gamma function.

\subsection{First moment}
\label{sec:size-mom1}
%
The first moment of $\mathbf{s}$ is given by Equation~(\ref{eq:e1}).
We may use Equations~(\ref{eq:s}) and (\ref{eq:pdf_r}) to write the
$n^{\mathrm{th}}$ term of $\mathbf{{e_1}}(\pmb{\theta}_r)$ as
%
\begin{align}
    {e_1}_n(\alpha, \beta)
     & =
    \resizebox{.75\hsize}{!}{$
            \begin{dcases}
                -\dfrac{\Gamma(\alpha + \beta)}{\Gamma(\alpha)\Gamma(\beta)}
                \int\limits_0^1
                \dfrac{
                    (c_0 + c_1 r)
                    r^{\alpha - 1}
                    (1 - r)^{\beta - 1}
                }{2 (1 + c_0 + c_1 r)}
                \mathrm{d} r
                 &
                \qquad
                l = m = 0    \\[2em]
                -\dfrac{\Gamma(\alpha + \beta)}{\Gamma(\alpha)\Gamma(\beta)}
                \int\limits_0^1
                \dfrac{(c_0 + c_1 r) \left( 2 + c_0 + c_1 r \right)r^{\alpha - 1}
                (1 - r)^{\beta - 1}}
                {2 \sqrt{2l + 1} (1 + c_0 + c_1 r)^{l + 1}}
                \mathrm{d} r
                 &
                \qquad
                l > 0, m = 0 \\[2em]
                0
                 &
                \qquad m \ne 0
                \quad.
            \end{dcases}
        $}
\end{align}
%
Thanks to the dependence of the spherical harmonic coefficients on only
powers of $r$ and $(1 + c_0 + c_1 r)$, the integrals above may be expressed in closed
form in terms of the hypergeometric function ${_2F_1}$:
%
\begin{align}
    {e_1}_n(\alpha, \beta)
     & =
    \resizebox{.75\hsize}{!}{$
            \begin{dcases}
                -
                \dfrac{1}{2(1 + c_0)}
                \bigg[c_0H_0^0 + c_1 H_0^1\bigg]
                 &
                \qquad
                l = m = 0    \\[2em]
                -\dfrac{1}{2\sqrt{2l + 1} (1 + c_0)^{l+1}}
                \bigg[
                    c_0(2+c_0)H_l^0
                    +
                    2c_1(1 + c_0)H_l^1
                    +
                    c_1^2 H_l^2
                    \bigg]
                 &
                \qquad
                l > 0, m = 0 \\[2em]
                0
                 &
                \qquad m \ne 0
            \end{dcases}
        $}
\end{align}
%
where we define
%
\begin{align}
    H_j^k & \equiv \left(\prod_{n=0}^{k-1} \lambda_n\right) G_j^k
    \quad.
\end{align}
%
with
%
\begin{align}
    \lambda_n & \equiv \dfrac{\alpha + n}{\alpha + \beta + n}
\end{align}
%
and
%
\begin{align}
    G_j^k & \equiv {_2F_1}\left(j + 1, \alpha + k; \alpha + \beta + k; -\dfrac{c_1}{1 + c_0}\right)
    \quad.
\end{align}
%
Note that for fixed $\alpha$, $\beta$, $c_0$, and $c_1$, we need only compute
$G_0^0$, $G_1^0$, $G_0^1$, and $G_1^1$ directly, since the remaining terms may be
obtained recursively:
%
\begin{proof}{test_hypgeo}
    \label{eq:Grec}
    \begin{split}
        G_j^k & =
        \bigg[
            \dfrac{(\alpha + \beta + k - j)(1 + c_0)}{j(1 + c_0 + c_1)}
            \bigg] G_{j - 2}^k
        \\[0.75em]
        &
        + \hspace{1.5pt}
        \bigg[
            1 - \dfrac{(\alpha + \beta + k - j)(1 + c_0) + (\alpha + k)c_1}
            {j(1 + c_0 + c_1)}
            \bigg]
        G_{j - 1}^k
        \\[1.5em]
        G_j^k & =
        \bigg[
            \frac{(\alpha + \beta + k - 2)(1 + c_0)}
            {(\alpha + \beta - j + k - 2)\lambda_{k - 1}c_1}
            \bigg]
        G_{j}^{k - 2}
        \\[0.75em]
        & + \hspace{1.5pt}
        \bigg[
            \frac{1}{\lambda_{k - 1}}
            -\frac{(\alpha + \beta + k - 2)(1 + c_0) + \beta c_1}
            {(\alpha + \beta - j + k - 2)\lambda_{k-1}c_1}
            \bigg]
        G_{j}^{k - 1}
        \quad.
    \end{split}
    \raisetag{7.5em}
\end{proof}
%

%
\vspace{1em}
%

\subsection{Second moment}
\label{sec:size-mom2}
%
The second moment of $\mathbf{s}$ is given by Equation~(\ref{eq:E1}).
We may again use Equations~(\ref{eq:s}) and (\ref{eq:pdf_r}) to write the
term at index $(n, n')$ of $\mathbf{{E_1}}(\pmb{\theta}_r)$ as
%
\begin{align}
    {E_1}_{n,n'} (\alpha, \beta) & =
    \resizebox{1.5\hsize}{!}{$
            \begin{dcases}
                \dfrac{\Gamma(\alpha + \beta)}{\Gamma(\alpha)\Gamma(\beta)}
                \int\limits_0^1
                \left(
                \dfrac{c_0 + c_1 r}{2 (1 + c_0 + c_1 r)}
                \right)^2
                \\
                \qquad\qquad\quad\quad\quad\quad\quad
                \times
                \,\,
                r^{\alpha - 1}
                (1 - r)^{\beta - 1}
                \mathrm{d} r
                 &
                \qquad
                \parbox[t]{\linewidth}{\vspace{-2.5em}$l = l' = 0, \\m = m' = 0$}
                \\[2em]
                %
                %
                %
                \dfrac{\Gamma(\alpha + \beta)}{\Gamma(\alpha)\Gamma(\beta)}
                \int\limits_0^1
                \left(
                \dfrac{(c_0 + c_1 r)}{2 (1 + c_0 + c_1 r)}
                \right)
                \\
                \qquad\qquad\quad\quad\quad\quad\quad
                \times
                %
                \left(
                \dfrac{(c_0 + c_1 r) \left( 2 + c_0 + c_1 r \right)}
                    {2 \sqrt{2l + 1} (1 + c_0 + c_1 r)^{l + 1}}
                \right)
                %
                \\[0.75em]
                \qquad\qquad\quad\quad\quad\quad\quad
                \times
                \,\,
                r^{\alpha - 1}
                (1 - r)^{\beta - 1}
                \mathrm{d} r
                 &
                \qquad
                \parbox[t]{\linewidth}{\vspace{-5em}$l > 0,        \\l' = 0, \\m = m' = 0$}
                \\[2em]
                %
                %
                %
                \dfrac{\Gamma(\alpha + \beta)}{\Gamma(\alpha)\Gamma(\beta)}
                \int\limits_0^1
                \left(
                \dfrac{(c_0 + c_1 r) \left( 2 + c_0 + c_1 r \right)}
                    {2 \sqrt{2l + 1} (1 + c_0 + c_1 r)^{l + 1}}
                \right)
                \\
                \qquad\qquad\quad\quad\quad\quad\quad
                \times
                %
                \left(
                \dfrac{(c_0 + c_1 r) \left( 2 + c_0 + c_1 r \right)}
                    {2 \sqrt{2l' + 1} (1 + c_0 + c_1 r)^{l' + 1}}
                \right)
                %
                \\[0.75em]
                \qquad\qquad\quad\quad\quad\quad\quad
                \times
                \,\,
                r^{\alpha - 1}
                (1 - r)^{\beta - 1}
                \mathrm{d} r
                 &
                \qquad
                \parbox[t]{\linewidth}{\vspace{-5em}$l > 0,        \\l' > 0, \\m = m' = 0$}
                \\[2em]
                %
                %
                %
                0
                 &
                \qquad m, m' \ne 0
                \quad,
            \end{dcases}
        $}
    \raisetag{15.5em}
\end{align}
%
where $l' = l'(n')$ and $m' = m'(n')$ are again given by Equation~(\ref{eq:lm})
and, by symmetry, the expression for $l' > 0, \, l = 0, \, m = m' = 0$ is the same as
in the second case, provided we make the substitution $l \rightarrow l'$.
%
These integrals again reduce to closed form expressions:
%
\begin{align}
    {E_1}_{n,n'} (\alpha, \beta) & =
    \resizebox{1.4\hsize}{!}{$
            \begin{dcases}
                \dfrac{1}{4(1+c_0)^2}
                \bigg(
                c_0^2 H_1^0
                +
                2 c_0 c_1 H_1^1
                +
                c_1^2 H_1^2
                \bigg)
                 &
                \qquad
                \parbox[t]{\linewidth}{\vspace{-1.5em}$l = l' = 0, \\m = m' = 0$}
                \\[2em]
                %
                %
                %
                \dfrac{1}{4\sqrt{2l + 1}(1+c_0)^{l+2}}
                \\[0.75em]
                \enspace\quad
                \times
                \bigg(
                c_0^2(2 + c_0) H_{l+1}^0
                + c_0 c_1 (4 + 3 c_0) H_{l+1}^1
                \\[0.75em]
                \qquad
                + (2 + 3 c_0) c_1^2 H_{l + 1}^2
                + c_1^3 H_{l + 1}^3
                \bigg)
                 &
                \qquad
                \parbox[t]{\linewidth}{\vspace{-5.5em}$l > 0,      \\l' = 0, \\m = m' = 0$}
                \\[2em]
                %
                %
                %
                \dfrac{1}{4\sqrt{(2l + 1)(2l' + 1)}(1+c_0)^{l+l'+2}}
                \\[0.75em]
                \enspace\quad
                \times
                \bigg[
                    c_0^2(2+c_0)^2 H_{l + l' + 1}^0
                    + 4 c_0 (1 + c_0) (2 + c_0) c_1 H_{l + l' + 1}^1
                    \\[0.75em]
                \qquad
                + 2 \big(2 + 3 c_0 (2 + c_0)\big)c_1^2 H_{l + l' + 1}^2
                + 4(1 + c_0)c_1^3 H_{l + l' + 1}^3
                \\[0.75em]
                \qquad
                + c_1^4 H_{l + l' + 1}^4
                \bigg]
                 &
                \qquad
                \parbox[t]{\linewidth}{\vspace{-6em}$l > 0,        \\l' > 0, \\m = m' = 0$}
                \\[2em]
                %
                %
                %
                0
                 &
                \qquad m, m' \ne 0
                \quad.
            \end{dcases}
        $}
    \raisetag{13.5em}
\end{align}
%

\section{The Contrast Integrals: \lowercase{$\mathbf{e_2}$} and $\mathbf{E_2}$}
\label{sec:contrast}
%
\subsection{Functional form}
\label{sec:contrast-function}
Since the spot function $\mathbf{s}$ (Equation~\ref{eq:s}) is normalized
to $-1$, we may simply scale it by the contrast $\xi$ to obtain a spot
whose peak intensity is $-\xi$.
Assuming the baseline (unspotted) stellar intensity is unity, $\xi$
is the fractional decrease in the brightness relative to the stellar
baseline: i.e., the spot contrast.
%

\subsection{PDF}
\label{sec:contrast-pdf}
We require our
spot contrast to be no larger than $1$ (corresponding to zero intensity
at the center of the spot when the baseline is unity) to enforce physical
intensities
everywhere. We have thus far implicitly assumed the spot is dark, but
in principle bright spots may also exist
(such as the plages observed on the Sun), corresponding to contrasts
$\xi < 0$. We therefore require a PDF with support over $(-\infty, 1)$.
Defining
%
\begin{align}
    b(\xi) \equiv 1 - \xi
\end{align}
%
to be the actual brightnesss of a unit-intensity stellar surface at
the center of the spot,
we can obtain the required support by modeling
$b \in (0, \infty)$ as a log-normal
random variable:
%
\begin{align}
    \label{eq:lognormal}
    p \big(b \, \big| \, \mu, \nu \big)
     & =
    \frac{1}{b\sqrt{2\pi\nu}}
    \exp\left[
        -\dfrac{\big(\ln b - \mu\big)^2}{2\nu}
        \right]
    \quad.
\end{align}
%
where $\mu$ and $\nu$ are the mean and variance in $b$, respectively.
The corresponding PDF in terms of $\xi$ is
%
\begin{align}
    \label{eq:lognormal}
    p \big(\xi \, \big| \, \pmb{\theta}_\xi \big)
     & =
    \frac{1}{(1 - \xi)\sqrt{2\pi\nu}}
    \exp\left[
        -\dfrac{\big(\ln (1 - \xi) - \mu\big)^2}{2\nu}
        \right]
    \quad.
\end{align}
%
where the hyperparameters are
%
\begin{align}
    \pmb{\theta}_\xi = \left(
    \mu \, \, \, \,
    \nu \right)^\top
    \quad.
\end{align}
%

\subsection{First moment}
\label{sec:contrast-mom1}
The first moment of the spot contrast distribution is given by
Equation~(\ref{eq:e2}), which may easily be solved analytically:
%
\begin{align}
    \mathbf{e_2}(\pmb{\theta}_b, \mathbf{e_1})
     & =
    \int_{-\infty}^1
    \xi \,
    \mathbf{e_1} \,
    p(\xi \, \big| \, \pmb{\theta}_{\xi}) \,
    \mathrm{d}\xi
    \nonumber
    \\
     & =
    \left(1 - \exp\left[ \mu + \frac{1}{2}\nu\right]\right) \mathbf{e_1}
    \quad.
\end{align}
%

\subsection{Second moment}
\label{sec:contrast-mom2}
The second moment of the spot contrast distribution is given by
Equation~(\ref{eq:E2}), which also evaluates to a simple closed form:
%
\begin{align}
    \mathbf{E_2}(\pmb{\theta}_b, \mathbf{E_1})
     & =
    \int_{-\infty}^1
    \xi^2 \,
    \mathbf{E_1} \,
    \mathbf{E_1}^\top \,
    p(\xi \, \big| \, \pmb{\theta}_{\xi}) \,
    \mathrm{d}\xi
    \nonumber
    \\
     & =
    \left(1 - 2\exp\bigg[ \mu + \frac{1}{2}\nu\bigg]
    + \exp\bigg[ 2\mu + 2\nu\bigg]\right)
    \mathbf{E_1} \,
    \mathbf{E_1}^\top
    \quad.
\end{align}



\bibliography{bib}
\end{document}